%!TEX root = main.tex



The resolution of wireless sensing is inversely proportional to the bandwidth. 
% The latest 802.11ac/11ax Wi-Fi can work in 160MHz bandwidth, or 6.2$ns$ timing resolution,
% however, it is still insignificant compared to Ultra-Wide Bandwidth (UWB).
%On the other hand, the COTS 802.11ac Wi-Fi can provide 8 spatial streams in one single station, while the UWB system, which is sketched upon the Vector Network Analyzer (VNA) or USRP, is lack of scalability.
Aiming to achieve finer resolution on COTS devices, 
Xie \etal~\cite{Xie2015Precise} and Zhu \etal~\cite{Zhu:2018fc} 
propose the spectrum stitching technology.
However, they both suffer from the standard Wi-Fi channelization, \ie,
very limited total bandwidth,
channel discontinuity in 5GHz band,
and the coarse 5/20MHz channel spacing.

To the best of our knowledge,
Atheros ar9300 series and Intel 5300 NICsare 
the only feasible platforms for spectrum stitching technology,
because the stitching depends on CSI measurement, 
which is publicly available only on these two platforms\cite{csitool, Xie2015Precise}.
Nevertheless, two NICs have different architects under the hood.
%
Intel 5300 adopts the \textit{fullMAC} architect, 
that the radio front end only communicates with the a piece of close-sourced firmware;
while the kernel driver, resembling a messenger, is blocked from the PHY layer control.
%
On the contrary, ar9300 series adopt the \textit{softMAC} design, \ie, firmware-free.
In this case, the kernel driver controls the every aspect of the hardware details, 
such as the one of our most interesting controls, the central frequency (CF) synthesizer.

Since both NICs support 2.4/5GHz dual-band Wi-Fi, 
their CF synthesizers, at least in hardware aspect, support much larger working range and finer resolution.
We failed on synthesizer tuning on Intel 5300, due to its \textit{fullMAC} architect.
Fortunately, on ar9300 NIC, we successfully override its CF synthesizer control and perform evaluations to find the true CF range boundary. The verified CF range boundary is summarized in Table~\ref{tab:cf_tuning_range}\footnote{The CF tuning resolution is a precise value reversely engineered from the driver; however, the CF ranges are explored by experiments and is relatively \textit{conservative}.}. 

Besides the tuning range, 
the CF switching speed is also blazingly fast. 
In same-band case, the switching costs less than 1$us$. 
In cross-band case, hardware reset is required 
and it costs less than 40$us$ on a dual-core laptop.
In such short time, the standard 600MHz bandwidth in 5GHz band can be scanned in just less than 3$ms$, which is much smaller than the coherently in typical indoor environment.




\begin{table}[tb]
	\caption{CF Tuning Boundary }
	\label{tab:cf_tuning_range}
	\centering

	\begin{tabular}{llll}
	\hline

	\hline
	\textbf{Band} & \textbf{Lower-Bound} & \textbf{Upper-Bound} & \textbf{Resolution} \\
	\hline
	2.4GHz & 2.2GHz & 2.9GHz & 250Hz \\
	\hline
	5GHz   & 4.4GHz & 6.1GHz & 1KHz \\
	\hline

	\hline
	\end{tabular}
\end{table}
